\documentclass[12pt]{article}

% --- 中文支持(XeLaTeX)---
\usepackage[UTF8]{ctex}

% --- 数学 ---
\usepackage{amsmath, amssymb}

% --- 版式/表格/图片 ---
\usepackage{geometry}
\usepackage{graphicx}
\usepackage{booktabs}
\usepackage{float}
\usepackage{fancyhdr}
\usepackage{cite}
\usepackage{tabularx}

% 图片路径
\graphicspath{{./}{docs/}}

% --- 超链接(尽量最后加载)---
\usepackage{hyperref}


\geometry{a4paper, margin=2.2cm}
\pagestyle{fancy}
\fancyhf{}
\fancyhead[C]{深度神经网络架构研究:PINN 与 DeepONet}
\fancyfoot[C]{\thepage}

\title{\textbf{PINN 正反问题中的架构设计与采样分布对精度与泛化的影响:基于 DeepXDE 的对比研究}}
\author{2025级\quad 数学\quad 刘倩妃}
\date{2026年1月10日}

\begin{document}
\maketitle

\section{实验环境}\label{sec:1}

\begin{itemize}
  \item 问题1: 使用 DeepXDE + PyTorch,默认浮点精度为 \texttt{float64},Adam 学习率 $10^{-3}$、迭代 20000 步,并切换 L-BFGS 进行二阶段优化。
  \item 问题2: 使用 DeepXDE + PyTorch,默认浮点精度为 \texttt{float32},Adam 学习率 $10^{-3}$、迭代 10000 步。
\end{itemize}

\section{问题1: PINN 求解 Lotka--Volterra 正问题}\label{sec:2}

\subsection{问题定义}\label{subsec:1}
Lotka--Volterra 模型常用于描述捕食-被捕食种群动力学。项目实现中对状态进行尺度化:设总体缩放常数 $U_B=200$,时间尺度 $R_B=20$,在区间 $t\in[0,1]$ 上求解。令网络输出为 $\hat y(t)=[\hat r(t),\hat p(t)]$(均为尺度化量),并通过自动微分计算导数。对应残差形式(以项目实现为准)可写为
\begin{equation}
\mathcal{R}_r(t)=\frac{d\hat r}{dt}-\frac{R_B}{U_B}\Big(2U_B\hat r-0.04(U_B\hat r)(U_B\hat p)\Big),\label{eq:equation2}
\end{equation}
\begin{equation}
\mathcal{R}_p(t)=\frac{d\hat p}{dt}-\frac{R_B}{U_B}\Big(0.02(U_B\hat r)(U_B\hat p)-1.06(U_B\hat p)\Big).\label{eq:equation3}
\end{equation}
初值条件为
\begin{equation}
\hat r(0)=\frac{100}{U_B},\qquad \hat p(0)=\frac{15}{U_B}.\label{eq:equation4}
\end{equation}

\subsection{PINN 训练目标:残差约束与初值约束}\label{subsec:2}
PINN 的典型损失由“方程残差 + 初值/边界误差”组成~\cite{pinn_raissi}。本实验中重点对比了\textbf{硬初值约束(hard IC)}与不同输入特征编码方式。对于 hard IC,项目采用输出变换将初值条件结构化地写入网络输出(以实现为准):
\begin{equation}
\tilde r(t)= y_1(t)\tanh(t)+\frac{100}{U_B},\qquad
\tilde p(t)= y_2(t)\tanh(t)+\frac{15}{U_B},\label{eq:equation5}
\end{equation}
其中 $y_1,y_2$ 为网络“未约束输出”。该设计的优点是:无论参数如何更新,$t=0$ 时总有 $\tanh(0)=0$,从而严格满足初值;同时 $\tanh(t)$ 在小 $t$ 区间近似线性,仍保留可训练自由度。

\subsection{网络结构与特征编码}\label{subsec:3}

实验对比包含三类结构(均为 6 层宽度 64 的主干思路):
\begin{itemize}
  \item \textbf{FNN:}标准全连接网络;
  \item \textbf{PFNN:}分支式并行结构(在实现中体现为部分层用列表表示的并行通道);
  \item \textbf{SkipFNN:}带残差/跳连的 MLP(提高优化稳定性与梯度流动效率)。
\end{itemize}

项目中对时间输入 $t$ 使用两种特征变换:
\begin{itemize}
  \item \textbf{sin-only:}拼接 $[t,\sin(t),\sin(2t),\dots,\sin(6t)]$(共 7 维);
  \item \textbf{Fourier 多尺度:}拼接 $[t,\sin(2\pi f t),\cos(2\pi f t)]$,频率 $f\in\{1,2,4,8\}$(共 9 维)。
\end{itemize}
直观上,Lotka--Volterra 的解呈现明显的周期性与尖峰式振荡结构。Fourier 特征更贴近“谱空间表达”~\cite{fourier_features},通常能缓解神经网络的低频偏置(spectral bias)并提升对高频/局部快速变化的拟合能力。

\subsection{采样策略与训练设置}\label{subsec:4}
本报告展示的图均对应 \textbf{hardIC + pseudo\_3k\_b2} 的配置(见图标题):
\begin{itemize}
  \item 训练域点数:\texttt{num\_domain=3000}(pseudo 分布);
  \item 边界/初值点数:\texttt{num\_boundary=2}(hard IC 下边界点更多是辅助稳定,而非必须满足);
  \item Adam 阶段:学习率 $10^{-3}$,训练到 $2\times 10^4$ 步量级后切换 L-BFGS 做二阶段拟合(图中 loss 横轴为 steps)。
\end{itemize}

\subsection{实验结果与分析}\label{subsec:5}

\subsubsection{Train/Test loss 的阶段性特征与异常现象}
 图\ref{fig:p1_loss_compare} 展示了 Adam 阶段 train/test loss 随步数的变化。可以观察到两个值得强调的现象:

一、在训练初期(数千步以内),所有方法 loss 都快速下降若干数量级,说明 PINN 能迅速捕捉到动力学的大致趋势;之后进入平台/震荡区间,优化主要在细节与高频结构上“抠精度”。

二、橙色虚线(\texttt{FNN\_tanh\_sinfeat\_hardIC\_\_pseudo\_3k\_b2 | test})在约 $2.5\times 10^4$ 步后出现明显上升并长期维持较高水平;与之对比,Fourier 特征相关模型的 test loss 能继续下降或保持稳定。这一现象说明:\textbf{同样的 hard IC 与采样点数下,特征编码会显著改变模型的可泛化性}——并且这种差异不仅体现在最终误差,也体现在训练后期的稳定性上。

\begin{figure}[H]
    \centering
    \includegraphics[width=0.90\textwidth]{images/problems/problem01/solution_v2_1.png}
    \caption{Adam 阶段 Train/Test loss 对比(hardIC + pseudo\_3k\_b2)。}
    \label{fig:p1_loss_compare}
\end{figure}

\subsubsection{预测曲线对比:Fourier 特征显著更稳健}
 图\ref{fig:p1_pred_compare} 给出 $r(t)$ 与 $p(t)$ 在 $t\in[0,1]$ 上的预测曲线。该曲线说明问题:


\begin{itemize}
  \item \textbf{Fourier 特征 FNN(绿色虚线)与 SkipFNN(紫色虚线)}基本与真值重合,峰值位置与幅值都较准确;
  \item \textbf{sin-only 特征 FNN(橙色虚线)}很快塌陷到接近 0 的水平,与真实的周期峰值相去甚远;
  \item \textbf{PFNN(红色虚线)}能生成周期结构,但在部分峰附近存在可见的相位/幅值偏差,表明并行结构并非对该任务最优,或其优化更敏感。
\end{itemize}

这说明,对于带有尖峰的周期动力学,仅用 $\sin(kt)$(且不含 cos 互补项、也不含 $2\pi$ 归一化频率)可能不足以提供稳定的“相位表达基底”;Fourier(sin+cos)多尺度特征能够提供更完整的周期基表示,从而让网络更容易逼近真实轨道。

\begin{figure}[H]
    \centering
    \includegraphics[width=0.98\textwidth]{images/problems/problem01/solution_v2_2.png}
    \caption{$r(t)$ 与 $p(t)$ 的预测对比(hardIC + pseudo\_3k\_b2)。}
    \label{fig:p1_pred_compare}
\end{figure}

\subsubsection{最优模型与误差量级}
 图\ref{fig:p1_best} 展示了最优配置:
\texttt{FNN\_tanh\_fourier\_hardIC\_\_pseudo\_3k\_b2},
其图中给出的整体相对误差为 \texttt{err\_all=6.620e-04}。从曲线看,$r(t)$ 与 $p(t)$ 在多个周期峰的峰值、谷值、回落段均能与真值紧密贴合,说明此配置在表达能力、优化稳定性与约束实现上达成了较好的平衡。

\begin{figure}[H]
  \centering
   \includegraphics[width=0.78\textwidth]{images/problems/problem01/solution_v2_3.png}
  \caption{最优模型预测效果(err\_all=6.620e-04)。}
  \label{fig:p1_best}
\end{figure}

\subsection{讨论:为何 Fourier + hardIC 更有效?}

\subsubsection{谱偏置与周期系统的“频域捷径”}
经典现象是:标准 MLP 在训练早期更容易学习低频成分(谱偏置)~\cite{spectral_bias}。而 Lotka--Volterra 的解虽然整体周期,但峰附近存在较强的局部高频变化。Fourier 特征将输入映射到更适合表示周期的空间,使网络更容易在较少参数更新下捕捉相位与频率结构,从而减少后期不稳定振荡或塌陷。

\subsubsection{硬约束降低可行域维度,提升后期稳定性}
hard IC 在结构上保证初值严格满足,相当于从优化角度减少了“先拟合初值再拟合残差”的竞争关系,使得优化更多专注于动力学残差。对敏感系统而言,初值的微小偏差会随时间放大;hard IC 能显著降低误差传播起点的不确定性,这对长程稳定预测尤其关键。
% hard IC 通过式\eqref{eq:hardic_p01}在结构上保证初值严格满足,相当于从优化角度减少了“先拟合初值再拟合残差”的竞争关系,使得优化更多专注于动力学残差。对敏感系统而言,初值的微小偏差会随时间放大;hard IC 能显著降低误差传播起点的不确定性,这对长程稳定预测尤其关键。

\subsubsection{结构选择:SkipFNN 的优势与 PFNN 的敏感性}
 从图\ref{fig:p1_pred_compare}可见 SkipFNN(紫色虚线)与 Fourier FNN(绿色虚线)都表现稳定。一般来说,残差连接有助于缓解深层网络训练中的梯度消失/鞍点停滞,使得在相同深度下更容易优化;而 PFNN 的并行结构在该实验中未体现出明显优势,可能原因包括:并行通道之间的特征分工未被充分利用,或超参数(宽度/深度/初始化)对 PFNN 更敏感。

\subsection{小节总结}
\begin{quote}
在相同采样与 hard IC 约束下,输入特征编码对 PINN 的训练稳定性与最终精度具有决定性影响;Fourier(sin+cos,多尺度)特征显著优于 sin-only 特征,并能与 FNN/SkipFNN 结构形成稳定组合。
\end{quote}

\section{问题2: DeepONet 学习算子(Antiderivative)}

\subsection{问题定义}
目标算子为
\begin{equation}
\mathcal{G}: f(x)\mapsto u(x)=\int_{0}^{x} f(s)\,ds,\qquad x\in[0,1].
\end{equation}
%项目使用训练/测试数据集:\href{https://yaleedu-my.sharepoint.com/personal/lu_lu_yale_edu/_layouts/15/onedrive.aspx}{\texttt{antiderivative\_aligned\_test.npz}},\href{https://yaleedu-my.sharepoint.com/personal/lu_lu_yale_edu/_layouts/15/onedrive.aspx}{\texttt{antiderivative\_aligned\_train.npz}})。
%训练数据集包含 $n_{\text{train}}=150$ 个函数样本,测试数据集包含 $n_{\text{test}}=1000$ 个函数样本;每个输入函数在 $m=100$ 个 sensors 位置采样,输出 $u(x)$ 在 $100$ 个 trunk 位置给出真值(full trunk)。

%\subsection{DeepONet 结构:Branch--Trunk 组合}
DeepONet 的基本形式为~\cite{deeponet_lu}
\begin{equation}
\hat u_\theta(x;f)=\sum_{k=1}^{p} b_\theta^{(k)}(f(x_1),\ldots,f(x_m))\; t_\theta^{(k)}(x),
\end{equation}
其中 Branch 网络将离散化的输入函数值映射为潜在表征向量,Trunk 网络将位置 $x$ 映射为基函数表征。本次实验使用 \texttt{deepxde.nn.tensorflow.deeponet模块} 实现,将一个 batch 的多函数输入与一个 trunk 点集进行笛卡尔积组合预测。

%\subsection{硬约束(hard IC):强制 $u(0)=0$}
对不定积分而言,天然边界条件为 $u(0)=0$。项目在 E6 中引入 hard IC,通过输出变换把该条件结构化写入:
\begin{equation}
% \label{eq:hardic_p02}
\tilde u(x)=x\cdot \hat u(x).
\end{equation}
由于 $x=0$ 时 $\tilde u(0)=0$ 恒成立,从而严格满足边界条件。这种约束同样具有“减少优化冲突、提高边界一致性”的作用,但它也会改变函数的有效表达空间(例如在 $x$ 很小时会放大相对误差敏感性),因此需要与采样与正则化配合。

%\subsection{Trunk 特征:缩放与 Fourier features}
在 E6 中,trunk 输入还做了两步处理:
\begin{itemize}
  \item 将 $x\in[0,1]$ 线性映射到 $[-1,1]$;
  \item 拼接 Fourier 特征:$[x,\sin(kx),\cos(kx)]$,$k=1,\dots,8$(附带比例系数)。
\end{itemize}
该设计增强了 trunk 基函数的表达能力,使其更容易生成复杂的函数形状;但在 trunk 采样稀疏时,也更容易出现插值不稳定。

\subsection{实验设计:E1--E6 的对比维度}
本次实验使用训练/测试数据集:\href{https://yaleedu-my.sharepoint.com/personal/lu_lu_yale_edu/_layouts/15/onedrive.aspx}{\texttt{antiderivative\_aligned}}。
训练数据集包含 $n_{\text{train}}=150$ 个函数样本,测试数据集包含 $n_{\text{test}}=1000$ 个函数样本;每个输入函数在 $m=100$ 个 sensors 位置采样,输出 $u(x)$ 在 $100$ 个 trunk 位置给出真值(full trunk)。
问题2的实验对比围绕四个关键因素:
\begin{enumerate}
  \item \textbf{sensors 数量:}100(full) vs 20;
  \item \textbf{trunk 数量:}100(full) vs 20;
  \item \textbf{trunk 采样分布:}uniform vs betaBoundary(更偏向边界附近采样);
  \item \textbf{Branch 架构与特征:}MLP / CNN branch / ResMLP branch + trunk Fourier + hardIC。
\end{enumerate}


表~\ref{tab:p2_cfg} 汇总对照实验 E1--E6 6种配置:

\begin{table}[H]
  \centering
  \caption{E1--E6 实验配置}
  \label{tab:p2_cfg}
  \setlength{\tabcolsep}{4pt} % 可调:减小列间距
  \renewcommand{\arraystretch}{1.15} % 可调:增大行距

  \begin{tabularx}{\linewidth}{@{} l l l l X @{}}
    \toprule
    实验 & sensors & trunk & trunk采样 & 网络与关键设置 \\
    \midrule
    E1 & 100 & 100 & uniform &
    MLP-MLP,latent=64,ReLU,no hardIC,no Fourier \\
    E2 & 20  & 100 & uniform &
    MLP-MLP,latent=64,ReLU,no hardIC \\
    E3 & 20  & 20  & uniform &
    MLP-MLP,latent=64,ReLU,no hardIC \\
    E4 & 20  & 20  & betaBoundary &
    MLP-MLP,latent=64,ReLU,no hardIC \\
    E5 & 20  & 20  & uniform &
    CNN Branch,latent=64,ReLU,no hardIC \\
    E6 & 20  & 20  & betaBoundary &
    ResMLP Branch,latent=128,ReLU,hardIC,trunk: scale+Fourier(8) \\
    \bottomrule
  \end{tabularx}
\end{table}

%\begin{table}[H]
%  \centering
%  \caption{Problem02:E1--E6 实验配置}
%   \label{tab:p2_cfg}
%  \begin{tabular}{@{}lllll@{}}
%    \toprule
%    实验 & sensors & trunk & trunk采样 & 网络与关键设置 \\
%    \midrule
%    E1 & 100 & 100 & uniform &
%    MLP-MLP,latent=64,ReLU,no hardIC,no Fourier \\
%    E2 & 20  & 100 & uniform &
%    MLP-MLP,latent=64,ReLU,no hardIC \\
%    E3 & 20  & 20  & uniform &
%    MLP-MLP,latent=64,ReLU,no hardIC \\
%    E4 & 20  & 20  & betaBoundary &
%    MLP-MLP,latent=64,ReLU,no hardIC \\
%    E5 & 20  & 20  & uniform &
%    CNN Branch,latent=64,ReLU,no hardIC \\
%    E6 & 20  & 20  & betaBoundary &
%    ResMLP Branch,latent=128,ReLU,hardIC,trunk: scale+Fourier(8) \\
%    \bottomrule
%  \end{tabular}\label{tab:table}
%\end{table}

所有实验采用 Adam 优化器、学习率 $10^{-3}$、训练 10000 步。以固定间隔抽样记录 train loss、test loss 以及一个在 trunk 点上的相对误差指标(mean rel L2)。

项目中使用的 mean relative L2(对每个函数样本先算相对 L2,再对样本取均值):
\begin{equation}
\mathrm{RelL2}=\frac{1}{N}\sum_{i=1}^{N}\frac{\|u_i-\hat u_i\|_2}{\|u_i\|_2+\varepsilon}.
\end{equation}
% 此外,在总结时还会计算 full grid 上的均值相对误差(与图\ref{fig:p2_fullgrid_error}一致展示)。
根据最终数据进一步统计 full grid 上的均值相对误差。


\subsection{实验结果与分析}

%\subsubsection{Train/Test loss 与 trunk 指标:多数实验“看起来都不错”}
% 图\ref{fig:p2_metric_curve} 汇总了 E1--E6 的三条曲线(train loss、test loss、trunk 指标)。可以看到:
图~\ref{fig:p2_metric_curve} 汇总了 E1--E6 的三条曲线(train loss、test loss、trunk 指标)。可以看到:
\begin{itemize}
  \item 所有实验在 0--1000 步内迅速下降,说明该算子学习任务在数据驱动下较易被拟合;
  \item 之后进入平台期,各方法的 test loss 与 trunk 指标差异并不巨大;
  \item 从 trunk 指标看,E6 在中后期甚至可能呈现较优水平(曲线更低)。
\end{itemize}
如果只依据这组三图,直觉上会认为:E6(更强的 branch、更强的 trunk 表达、更硬的边界约束)应当是“更强模型”。

\begin{figure}[H]
    \centering
    \includegraphics[width=0.90\textwidth]{images/problems/problem02/plot1_train_test_metric_compare.png}
    \caption{E1--E6 的 Train loss / Test loss / trunk 指标对比}
    \label{fig:p2_metric_curve}
\end{figure}

%\subsubsection{Running-min 训练损失:E6 的拟合能力最强}
% 图\ref{fig:p2_runningmin} 展示 train loss 的 running-min 曲线。E6 在全程显著低于其他实验(最低可到 $10^{-7}$ 量级附近),说明它在训练目标上具有更强的优化与拟合能力。一般而言,这对应于更大的模型容量(latent=128)、更强的特征表达(trunk Fourier)以及 hardIC 对边界误差的消除。
图~\ref{fig:p2_runningmin} 展示 train loss 的 running-min 曲线。E6 在全程显著低于其他实验(最低可到 $10^{-7}$ 量级附近),说明它在训练目标上具有更强的优化与拟合能力。一般而言,这对应于更大的模型容量(latent=128)、更强的特征表达(trunk Fourier)以及 hardIC 对边界误差的消除。


\begin{figure}[H]
  \centering
   \includegraphics[width=0.90\textwidth]{images/problems/problem02/plot2_loss_history_runningmin_compare.png}
  \caption{训练损失 running-min 对比}
   \label{fig:p2_runningmin}
\end{figure}

%\subsubsection{Full grid 真值对比:E6 出现明显振荡(核心现象)}
% 图\ref{fig:p2_true_pred} 在同一个测试样本上对比 full grid 的 $\hat u(x)$ 曲线。
图~\ref{fig:p2_true_pred} 在同一个测试样本上对比 full grid 的 $\hat u(x)$ 曲线。

结果非常关键:E1--E5 的预测几乎与真值重合(只有轻微偏差),而 E6(粉色虚线)出现明显的高频振荡、尖峰和局部偏离。

这说明:\textbf{E6 可能在训练所采样的 trunk 点上拟合得很好,但在点与点之间插值/外推时产生了非物理的高频波动}。
由于不定积分 $u(x)$ 本应相对平滑(若 $f$ 有界),这种振荡通常不是目标函数真实结构,而是模型自由度过大、采样过稀、缺少平滑先验共同作用的结果。

\begin{figure}[H]
    \centering
    \includegraphics[width=0.90\textwidth]{images/problems/problem02/plot3_true_vs_pred_compare.png}
    \caption{full grid 上 True vs Pred 对比}
    \label{fig:p2_true_pred}
\end{figure}

%\subsubsection{Full-grid 平均相对误差:E6 显著劣化}
% 图\ref{fig:p2_fullgrid_error} 给出 full grid 的 mean relative L2 error。
图~\ref{fig:p2_fullgrid_error} 给出 full grid 的 mean relative L2 error。
可以看到 E6 的误差条远高于其他实验(约 $0.4$ 量级),而 E1--E5 都处于较小水平。

因此,可以得出:
\begin{quote}
\textbf{训练损失或稀疏 trunk 点上的 test metric 并不必然等价于全域泛化质量;当模型表达过强而 trunk 采样稀疏时,可能出现“点上对、点间错”的高频振荡过拟合。}
\end{quote}

\begin{figure}[H]
    \centering
    \includegraphics[width=0.78\textwidth]{images/problems/problem02/solution_v2_1_1.png}
    \caption{full grid mean relative L2 error 对比}
    \label{fig:p2_fullgrid_error}
\end{figure}

%\subsection{机制讨论:E6 为什么会“训练更好但泛化更差”?}
E6 为什么会训练更好但泛化更差?

%\subsubsection{(1)trunk Fourier 特征提升表达,也提升了振荡风险}
trunk Fourier 特征提升表达,也提升了振荡风险。Fourier 特征等价于给 trunk 提供了一组高频基函数。若 trunk 点数足够多,它可以更好逼近复杂函数;但当 trunk 仅有 20 个点时,模型可能利用高频自由度在这些点上“对齐”,同时在点间产生快速摆动,从而导致 full grid 误差显著上升。

%\subsubsection{(2)betaBoundary 采样强调边界,可能弱化内部约束}
betaBoundary 采样强调边界,可能弱化内部约束。betaBoundary 的直观目标是让模型更重视边界附近的精度(对积分问题边界确实重要)。但当 trunk 总点数很少时,过度偏向边界会减少内部区间的有效约束密度,使内部插值更不稳定。

%\subsubsection{(3)hardIC 强化边界一致性,但不等价于“整体平滑先验”}
% hardIC 仅保证 $u(0)=0$(式\eqref{eq:hardic_p02}),它并不直接抑制 $u(x)$ 的振荡。换言之,hardIC 是“边界正确”的保证,但不是“形状合理”的保证。若没有额外的平滑正则或更密集的 trunk 覆盖,高频自由度仍可能导致不合理形状。
hardIC 强化边界一致性,但不等价于“整体平滑先验”。hardIC 仅保证 $u(0)=0$(式\$),它并不直接抑制 $u(x)$ 的振荡。换言之,hardIC 是“边界正确”的保证,但不是“形状合理”的保证。若没有额外的平滑正则或更密集的 trunk 覆盖,高频自由度仍可能导致不合理形状。


%\subsection{改进建议:如何把 E6 从“翻车”变成“可控优势”?}

为使更强表达(ResMLP + Fourier + hardIC)的优势真正转化为 full grid 的收益,分析得出如下改进方向:

%\subsubsection{(A)训练监控:引入更可信的验证指标}
\begin{itemize}
  \item 在训练过程中定期用更密集的 trunk(例如 100 或 200 点)计算验证误差,而不是只用 20 个 trunk 点;
  \item 在每个 step 随机重采样 trunk 点(stochastic trunk sampling),使模型无法只记住固定点位。
  \item 对积分问题,边界与内部都重要。可以采用混合采样:
    \begin{equation}
        x_{\text{trunk}} \sim \lambda\cdot \text{Uniform}(0,1) + (1-\lambda)\cdot \text{BetaBoundary},\label{eq:equation}
    \end{equation}
    在保证边界覆盖的同时不牺牲内部密度。
  \item 对 trunk Fourier 频率做截断(减小 $K$ 或降低 Fourier scale);
  \item 对输出曲线加入平滑正则(例如对离散网格上的二阶差分 $\|\Delta^2 u\|$ 做惩罚);
  \item 更贴合算子结构的方式:利用 $u'(x)=f(x)$,在训练中加入导数一致性项(软约束):
  \[
  \mathcal{L}_{\text{der}}=\| \partial_x \hat u(x) - f(x)\|^2,
  \]
  该项能够直接抑制“积分结果出现高频波动但导数不匹配”的情况。
\end{itemize}

%\subsubsection{(B)采样策略:uniform 与 boundary 混合}
%\begin{itemize}
%
%\end{itemize}

%\subsubsection{(C)正则化与结构先验:抑制不必要的高频}
%\begin{itemize}
%
%\end{itemize}

\subsection{小节总结}
%问题2,而是识别出一个常见但容易被忽略的风险:
\begin{quote}
\textbf{在算子学习/函数学习任务中,如果验证仅发生在稀疏点集上,强表达模型可能通过高频振荡在点上取得低误差,从而掩盖真实的全域泛化失败;full grid 评估与平滑先验是必要的。}
\end{quote}

\section{结论}
综合两个任务,可以得到一条贯穿性的经验规律:

\begin{enumerate}
  \item \textbf{表达方式要与问题结构匹配:}周期/振荡系统(问题1)更适合 Fourier 类特征;算子学习的 trunk(问题2)引入 Fourier 可提升容量,但需同时提升采样密度或引入平滑先验。
  \item \textbf{硬约束是“边界正确”的工具,而非“全域正确”的保证:}hard IC 能显著减少边界误差与训练冲突,但仍可能出现点间振荡或形状不合理,需要采样与正则协同。
  \item \textbf{评价指标必须覆盖真实使用场景:}问题2表明仅看 train/test loss 或稀疏 trunk metric 可能被误导;full grid 评估更能反映实际泛化质量。
\end{enumerate}

\begin{thebibliography}{99}
\bibitem{pinn_raissi}
M.~Raissi, P.~Perdikaris, and G.~E.~Karniadakis,
Physics-informed neural networks: A deep learning framework for solving forward and inverse problems involving nonlinear partial differential equations,
\emph{Journal of Computational Physics}, 2019.

\bibitem{deeponet_lu}
L.~Lu, P.~Jin, and G.~E.~Karniadakis,
Learning nonlinear operators via DeepONet based on the universal approximation theorem of operators,
\emph{Nature Machine Intelligence}, 2021.

\bibitem{fourier_features}
M.~Tancik et al.,
Fourier Features Let Networks Learn High Frequency Functions in Low Dimensional Domains,
\emph{NeurIPS}, 2020.

\bibitem{spectral_bias}
R.~Rahaman et al.,
On the Spectral Bias of Neural Networks,
\emph{ICML}, 2019.
\end{thebibliography}

\end{document}
